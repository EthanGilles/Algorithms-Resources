%\documentclass[10pt,a4paper]{article}
\documentclass[landscape, 10pt,a4paper]{report}
% Packages
\usepackage{fancyhdr}           % For header and footer
\usepackage{multicol}           % Allows multicols in tables
\usepackage{tabularx}           % Intelligent column widths
\usepackage{tabulary}           % Used in header and footer
\usepackage{hhline}             % Border under tables
\usepackage{graphicx}           % For images
\usepackage{xcolor}             % For hex colours
%\usepackage[utf8x]{inputenc}    % For unicode character support
\usepackage[T1]{fontenc}        % Without this we get weird character replacements
\usepackage{colortbl}           % For coloured tables
\usepackage{setspace}           % For line height
\usepackage{lastpage}           % Needed for total page number
\usepackage{seqsplit}           % Splits long words.
%\usepackage{opensans}          % Can't make this work so far. Shame. Would be lovely.
\usepackage[normalem]{ulem}     % For underlining links
% Most of the following are not required for the majority
% of cheat sheets but are needed for some symbol support.
\usepackage{amsmath}            % Symbols
\usepackage{MnSymbol}           % Symbols
\usepackage{wasysym}            % Symbols
%\usepackage[english,german,french,spanish,italian]{babel}              % Languages

\usepackage[margin=1.65in]{geometry}

\usepackage{pdflscape}
\usepackage{adjustbox}

% Document Info
\author{nhatlong0605}
\pdfinfo{
  /Title (git-cheat.pdf)
  /Creator (James Quinlan)
  /Author (Ethan Gilles)
  /Subject (Git/Github Commands)
}

% Lengths and widths
\addtolength{\textwidth}{7cm}
\addtolength{\textheight}{3cm}  % -1
\addtolength{\hoffset}{-3.2cm}
\addtolength{\voffset}{-3cm}
\setlength{\tabcolsep}{0.3cm} % Space between columns og=0.2
\setlength{\headsep}{-12pt} % Reduce space between header and content
\setlength{\headheight}{85pt} % If less, LaTeX automatically increases it
\renewcommand{\footrulewidth}{0pt} % Remove footer line
\renewcommand{\headrulewidth}{0pt} % Remove header line
\renewcommand{\seqinsert}{\ifmmode\allowbreak\else\-\fi} % Hyphens in seqsplit
% This two commands together give roughly
% the right line height in the tables
\renewcommand{\arraystretch}{1.3}
\onehalfspacing

% Commands
\newcommand{\SetRowColor}[1]{\noalign{\gdef\RowColorName{#1}}\rowcolor{\RowColorName}} % Shortcut for row colour
\newcommand{\mymulticolumn}[3]{\multicolumn{#1}{>{\columncolor{\RowColorName}}#2}{#3}} % For coloured multi-cols
\newcolumntype{x}[1]{>{\raggedright}p{#1}} % New column types for ragged-right paragraph columns
\newcommand{\tn}{\tabularnewline} % Required as custom column type in use

% Font and Colours
\definecolor{LogoBack}{HTML}{ffffff}

\definecolor{HeadBackground}{HTML}{336699}
\definecolor{FootBackground}{HTML}{9fa1a4}
\definecolor{TextColor}{HTML}{231f20}
\definecolor{DarkBackground}{HTML}{163e70} % 1D41A3
\definecolor{LightBackground}{HTML}{D8DEEF}
\renewcommand{\familydefault}{\sfdefault}
\color{TextColor}





% Header and Footer
\pagestyle{fancy}
\fancyhead{} % Set header to blank
\fancyfoot{} % Set footer to blank
\fancyhead[L]{
\noindent
\begin{multicols}{3}
\begin{tabulary}{5.8cm}{C}
    \SetRowColor{LogoBack}
    \vspace{-7pt}
    {\parbox{\dimexpr\textwidth-2\fboxsep\relax}{\noindent
        \hspace*{-6pt}\includegraphics[width=5.8cm]{logo.jpg}\\Department of Computer Sciences}
    }
\end{tabulary}
\columnbreak
\begin{tabulary}{11cm}{L}
  \vspace{-2pt}\large{\bf{\textcolor{DarkBackground}{\textsf{Git \& GitHub Commands}}}} \\
    \normalsize{ \textcolor{DarkBackground}{Department of Computer Sciences}}
\end{tabulary}
\columnbreak
\begin{tabulary}{11cm}{L}
    \vspace{-2pt}\large{\bf{\textcolor{DarkBackground}{\textsf{Git \& GitHub Commands}}}} \\
    \normalsize{ \textcolor{DarkBackground}{Department of Computer Sciences}}
\end{tabulary}
\end{multicols}}






\fancyfoot[L]{ \footnotesize
\noindent
\begin{multicols}{3}
\begin{tabulary}{5.8cm}{LL}
  \SetRowColor{FootBackground}
  \mymulticolumn{2}{p{5.377cm}}{\bf\textcolor{white}{Programs}}  \\
  \vspace{-2pt}BS Computer Science \\
 MS Data Science\\
  \end{tabulary}
\vfill
\columnbreak
\begin{tabulary}{5.8cm}{L}
  \SetRowColor{FootBackground}
  \mymulticolumn{1}{p{5.377cm}}{\bf\textcolor{white}{Cheat Sheet}}  \\
   \vspace{-2pt}Published ??th August, 2024.\\
   Updated 24th July, 2024.\\
\end{tabulary}
\vfill
\columnbreak
\begin{tabulary}{5.8cm}{L}
  \SetRowColor{FootBackground}
  \mymulticolumn{1}{p{5.377cm}}{\bf\textcolor{white}{Contact}}  \\
  \SetRowColor{white}
  \vspace{-5pt}
  %\includegraphics[width=48px,height=48px]{dave.jpeg}
  Dr. James Quinlan\\
  Chair, Dept. of Computer Science\\
\end{tabulary}
\end{multicols}}








\begin{document}
\raggedright
\raggedcolumns
% \begin{landscape}
% Set font size to small. Switch to any value
% from this page to resize cheat sheet text:
% www.emerson.emory.edu/services/latex/latex_169.html
\footnotesize % Small font.

\begin{multicols*}{3}

\begin{tabularx}{8.4cm}{X}
\SetRowColor{DarkBackground}
\mymulticolumn{1}{x{8.4cm}}{\bf\textcolor{white}{Setup}}  \tn

\SetRowColor{LightBackground}
\mymulticolumn{1}{x{8.4cm}}{\verb|git config ---global user.name "[username]"| \newline 
Set the Username Git will use to authenticate} \tn 

\SetRowColor{white}
\mymulticolumn{1}{x{8.4cm}}{\verb|git config ---global user.email "[email]"| \newline
Set the Email Git will use to authenticate} \tn 

\SetRowColor{LightBackground}
\mymulticolumn{1}{x{8.4cm}}{\verb|git config ---global color.ui auto| \newline 
Set automatic command line coloring for Git} \tn 

\SetRowColor{white}
\mymulticolumn{1}{x{8.4cm}}{\verb|git init | \newline
Initialize the current directory as a git repository} \tn 

\SetRowColor{LightBackground}
\mymulticolumn{1}{x{8.4cm}}{\verb|git clone [url]| \newline 
Download a local copy of a repository to the machine} \tn 

\hhline{>{\arrayrulecolor{DarkBackground}}-}
\end{tabularx}
\par\addvspace{0.9em}


\begin{tabularx}{8.4cm}{X}
\SetRowColor{DarkBackground}
\mymulticolumn{1}{x{8.4cm}}{\bf\textcolor{white}{Staging and Snapshots}}  \tn

\SetRowColor{white}
\mymulticolumn{1}{x{8.4cm}}{\verb|git status| \newline 
Show modified files that are staged for commit} \tn 

\SetRowColor{LightBackground}
\mymulticolumn{1}{x{8.4cm}}{\verb|git add [file/directory]| \newline
Stage a file for the next commit} \tn 

\SetRowColor{white}
\mymulticolumn{1}{x{8.4cm}}{\verb|git reset [file]| \newline 
unstage a file for commit while keeping the changes} \tn 

\SetRowColor{LightBackground}
\mymulticolumn{1}{x{8.4cm}}{\verb|git diff| \newline
Changes in files that are not staged for commit} \tn 

\SetRowColor{white}
\mymulticolumn{1}{x{8.4cm}}{\verb|git diff ---staged| \newline 
Changes in files that are staged for commit but not yet committed} \tn 

\SetRowColor{LightBackground}
\mymulticolumn{1}{x{8.4cm}}{\verb|git commit -m "[commit message]"| \newline
Commit your staged content as a new commit snapshot} \tn 

\SetRowColor{white}
\mymulticolumn{1}{x{8.4cm}}{\verb|git commit ---amend| \newline 
Add staged content to a commit snapshot that already exists} \tn 

\hhline{>{\arrayrulecolor{DarkBackground}}-}
\end{tabularx}
\par\addvspace{0.9em}



\begin{tabularx}{8.4cm}{X}
\SetRowColor{DarkBackground}
\mymulticolumn{1}{x{8.4cm}}{\bf\textcolor{white}{Initializing a Repository}}  \tn

\SetRowColor{white}
\mymulticolumn{1}{x{8.4cm}}{\verb|git init | \newline
Initialize the current directory as a git repository} \tn 

\SetRowColor{LightBackground}
\mymulticolumn{1}{x{8.4cm}}{\verb|git clone [url]| \newline 
Download a local copy of a repository to the machine} \tn 

\hhline{>{\arrayrulecolor{DarkBackground}}-}
\end{tabularx}



\begin{tabularx}{8.4cm}{X}
\SetRowColor{DarkBackground}
\mymulticolumn{1}{x{8.4cm}}{\bf\textcolor{white}{File Path Changes}}  \tn

\SetRowColor{LightBackground}
\mymulticolumn{1}{x{8.4cm}}{\verb|git rm [file]| \newline 
Delete the file from the project and stage the removal for commit} \tn 

\SetRowColor{white}
\mymulticolumn{1}{x{8.4cm}}{\verb|git mv [existing-path] [new-path]| \newline
Change an existing file path and stage the move} \tn 

\SetRowColor{LightBackground}
\mymulticolumn{1}{x{8.4cm}}{\verb|git log --stat -M| \newline 
Show all commit logs with an indication of any paths that changed} \tn 

\hhline{>{\arrayrulecolor{DarkBackground}}-}
\end{tabularx}
\par\addvspace{0.9em}



\begin{tabularx}{8.4cm}{X}
\SetRowColor{DarkBackground}
\mymulticolumn{1}{x{8.4cm}}{\bf\textcolor{white}{Branches and Merging}}  \tn

\SetRowColor{white}
\mymulticolumn{1}{x{8.4cm}}{\verb|git branch| \newline 
List all branches. The active branch is marked with a `*'} \tn 

\SetRowColor{LightBackground}
\mymulticolumn{1}{x{8.4cm}}{\verb|git branch [branch-name]| \newline
Create a new branch at the current commit} \tn 

\SetRowColor{white}
\mymulticolumn{1}{x{8.4cm}}{\verb|git branch -d [branch-name]| \newline
Delete the specified branch at the current commit} \tn 

\SetRowColor{LightBackground}
\mymulticolumn{1}{x{8.4cm}}{\verb|git remote add upstream URL.git| \newline
Add `upstream' repository to sync changes made in a forked repo} \tn 

\SetRowColor{white}
\mymulticolumn{1}{x{8.4cm}}{\verb|git checkout [branch-name]| \newline 
Switch branches and check it out into the working directory} \tn 

\SetRowColor{LightBackground}
\mymulticolumn{1}{x{8.4cm}}{\verb|git merge [branch]| \newline
Merge the selected branch's history into the current branch} \tn 

\hhline{>{\arrayrulecolor{DarkBackground}}-}
\end{tabularx}
\par\addvspace{0.9em}



\begin{tabularx}{8.4cm}{X}
\SetRowColor{DarkBackground}
\mymulticolumn{1}{x{8.4cm}}{\bf\textcolor{white}{Inspect and Compare}}  \tn

\SetRowColor{LightBackground}
\mymulticolumn{1}{x{8.4cm}}{\verb|git log| \newline 
Show all commits in the current branch's history} \tn 

\SetRowColor{white}
\mymulticolumn{1}{x{8.4cm}}{\verb|git log [branchB]..[branchA]| \newline 
Show the commits on branchA that are not on branchB} \tn 

\SetRowColor{LightBackground}
\mymulticolumn{1}{x{8.4cm}}{\verb|git log ---follow [file]| \newline
Show the commits that changed `file', even accross renames} \tn 

\SetRowColor{white}
\mymulticolumn{1}{x{8.4cm}}{\verb|git diff [branchB]..[branchA]| \newline 
Show the difference of what is in branchA but not in branchB} \tn 

\SetRowColor{LightBackground}
\mymulticolumn{1}{x{8.4cm}}{\verb|git show [SHA]| \newline
Show any object in Git in human-readable format} \tn 

\hhline{>{\arrayrulecolor{DarkBackground}}-}
\end{tabularx}
\par\addvspace{0.9em}



\begin{tabularx}{8.4cm}{X}
\SetRowColor{DarkBackground}
\mymulticolumn{1}{x{8.4cm}}{\bf\textcolor{white}{Share and Update}}  \tn

\SetRowColor{white}
\mymulticolumn{1}{x{8.4cm}}{\verb|git remote add [alias] [URL]| \newline 
Add a Git URL as an alias} \tn 

\SetRowColor{LightBackground}
\mymulticolumn{1}{x{8.4cm}}{\verb|git fetch [alias]| \newline 
Fetch down all the branches from that Git remote} \tn 

\SetRowColor{white}
\mymulticolumn{1}{x{8.4cm}}{\verb|git merge [alias]/[branch]| \newline
Merge remote branch into your current branch} \tn 

\SetRowColor{LightBackground}
\mymulticolumn{1}{x{8.4cm}}{\verb|git push [alias] [branch]| \newline 
Transmit local branch commits to the remote repository branch} \tn 

\SetRowColor{white}
\mymulticolumn{1}{x{8.4cm}}{\verb|git pull| \newline
Fetch and merge any commits from the tracking remote branch} \tn 

\hhline{>{\arrayrulecolor{DarkBackground}}-}
\end{tabularx}
\par\addvspace{1.3em}



\begin{tabularx}{8.4cm}{X}
\SetRowColor{DarkBackground}
\mymulticolumn{1}{x{8.4cm}}{\bf\textcolor{white}{Version Control}}  \tn

\SetRowColor{white}
\mymulticolumn{1}{x{8.4cm}}{\verb|git rebase [branch]| \newline 
Apply any commits of current branch ahead of the specified one} \tn 

\SetRowColor{LightBackground}
\mymulticolumn{1}{x{8.4cm}}{\verb|git reset ---hard [commit]| \newline 
Clear staging area, rewrite working tree from the specified commit} \tn 

\SetRowColor{white}
\mymulticolumn{1}{x{8.4cm}}{\verb|git checkout --- [file]| \newline 
Discard the changes to `file' in the working directory} \tn 

\SetRowColor{LightBackground}
\mymulticolumn{1}{x{8.4cm}}{\verb|git checkout [commit id]| \newline 
Revert to a previous commit} \tn 

\SetRowColor{white}
\mymulticolumn{1}{x{8.4cm}}{\verb|git revert [commit id]| \newline 
Undo the specified commit} \tn 

\hhline{>{\arrayrulecolor{DarkBackground}}-}
\end{tabularx}
\par\addvspace{1.3em}



\begin{tabularx}{8.4cm}{X}
\SetRowColor{DarkBackground}
\mymulticolumn{1}{x{8.4cm}}{\bf\textcolor{white}{Tokens}}  \tn

\SetRowColor{LightBackground}
\mymulticolumn{1}{x{8.4cm}}{\verb|git remote set-url origin| \newline
\verb|https://<NEW_TOKEN>@github.com/username/repo.git| \newline 
Set the remote repository using a Personal Access Token} \tn 

\SetRowColor{white}
\mymulticolumn{1}{x{8.4cm}}{Tokens can also be used to authenticate in the command line, for example: \newline
  \verb|$ git clone https://github.com/USERNAME/REPO.git| \newline
  \verb|Username: YOUR-USERNAME| \newline
  \verb|Password: YOUR-PERSONAL-ACCESS-TOKEN|} \tn 


\hhline{>{\arrayrulecolor{DarkBackground}}-}
\end{tabularx}
\par\addvspace{0.9em}

% That's all folks
\end{multicols*}
 % \end{landscape}
\end{document}
